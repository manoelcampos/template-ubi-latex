\documentclass[10pt,twoside]{estiloUBI}




\include{formatacaoUBI}

\usepackage{fontspec}
\setmainfont{Trebuchet MS}


%%Comentar a linha seguinte se escrever a tese em inglês
\portugues


%%Para índice remissivo
\makeindex


%%Escolher tipo de letra a usar:
%\usepackage{lmodern}												%Latin modern
%\usepackage{palatino}												%Palatino
%\usepackage{times}												%Times
%\renewcommand{\rmdefault}{trebuchet}										%Trebuchet (caso esteja instalado)


%%O comando seguinte insere o nome da tese no cabeçalho das páginas (comentar se não for pretendido)
\cabecalho{Inserir título da tese aqui (opcional)}



\begin{document}

%%O comando seguinte insere o espaçamento de 1.5 linhas
\onehalfspacing

%%Página de rosto
\pagenumbering{roman}
\begin{titlepage}
\begin{center}

\begin{flushleft}
 \includegraphics[height=2.22cm]{logo}\\
\rostoubi UNIVERSIDADE DA BEIRA INTERIOR\\
\rostofac Engenharia\\
\end{flushleft}

\vspace{7.6cm}

\rostotitulo \textbf{<Título da Tese>} \\
\rostosubtit \textbf{<Sub-Título da Tese>}\\

\vspace{1.8cm}

\rostonomes \textbf{<Nome do autor>}\\

\vspace{1.4cm}

\rostooutros Tese para obtenção do Grau de Doutor em\\
\rostonomes \textbf{<Designação do Curso>}\\
\rostooutros (3º ciclo de estudos)\\

\vspace{3.3cm}

\rostooutros Orientador: Prof. Doutor Nome\\
Co-orientador: Prof. Doutor Nome\\

\vspace{1.4cm}

\rostooutros \textbf{Covilhã, Junho de 2010}

\end{center}
\end{titlepage}



%\dominitoc


%%Numeração das páginas
\pagestyle{fancy}


%%O comando a seguir gera uma página após a de rosto com cabeçalho e rodapé
\cleardoublepage

%%O comando a seguir permite que as costas da página de rosto não inclua cabeçalho mas rodapé (escolher entre este e outro)
%\newpage\mbox{}\thispagestyle{plain}\fancyhead{}


%%Dedicatória
\newpage 
\section*{\titulos{Dedicatória}}
\vspace{0.5cm}
Inserir dedicatória (opcional)
\cleardoublepage
%\newpage 	
%\mbox{}
%\vfil
%\begin{center}
%Dedicated to...
%\end{center}
%\vfil
%\eject
%\cleardoublepage


%%Agradecimentos 
\newpage 	
\section*{\titulos{Agradecimentos}}
\vspace{0.5cm}
Agradecer a quem de direito (opcional)
\cleardoublepage


%%Prefácio 
\newpage 	
\section*{\titulos{Prefácio}}
\vspace{0.5cm}
Opcional
\cleardoublepage


%%Resumo+palavras-chave
\newpage 	
\section*{\titulos{Resumo}}
\vspace{0.5cm}
Resumo do trabalho
 
\vspace{2.2cm}
{\titulos{Palavras-chave}}

\vspace{0.8cm}
Inserir palavras-chave
\cleardoublepage


%%Resumo alargado 
\newpage 	
\section*{\titulos{Resumo alargado}}
\vspace{0.5cm}
Unicamente para teses em língua estrangeira
\cleardoublepage


%%abstract+keywords
\newpage 	
\section*{\titulos{Abstract}}
\vspace{0.5cm}
Abstract in English

\vspace{2.2cm}
{\titulos{Keywords}}
 
\vspace{0.8cm}
Keywords in English
\cleardoublepage


%%Índice
\tableofcontents


%%Lista de figuras
\listoffigures
\cleardoublepage	


%%Lista de tabelas
\listoftables
\cleardoublepage


%%Abreviaturas
\newpage
\section*{\titulos{Lista de Acrónimos}}
\vspace{0.5cm}
  \begin{tabularx}{\linewidth}{c p{0.5cm} Y}
 	UBI & & Universidade da Beira Interior\cr
 	MPSOCD & & Multi-objective Particle Swarm Optimization Crowding Distance
 	\end{tabularx}
 \cleardoublepage
  

%% Os capitulos são inseridos a partir daqui 
 
\mainmatter

\chapter{Introdução} \label{chap:int}

%% Para fazer um mini indice do capitulo abrir o ficheiro ``formatacaoUBI.tex" e procurar ''%% O código seguinte permite gerar um mini indice de capitulo (não referido no despacho reitoral)''
%\minitoc


\section{Objectivos}
Este documento pretende servir como modelo\index{modelo} para teses a apresentar na Universidade da Beira Interior (UBI). Para mais informações sobre o {\LaTeX} pode consultar \cite{short} ou \cite{eprojects}.

\section{Secção 2} \label{sec2}
Lorem ipsum dolor sit amet, consectetur adipiscing elit. Praesent at magna viverra neque bibendum pellentesque. Morbi ullamcorper auctor turpis vitae mollis. Fusce elementum mauris eu magna tristique vel aliquet erat iaculis. Donec sed augue mi. Aenean commodo lorem ac nulla iaculis rhoncus. Mauris facilisis, ante in molestie bibendum, lorem augue vehicula metus, ac auctor turpis quam nec purus. Nam malesuada accumsan neque, quis vulputate nibh dapibus vitae. Vestibulum eu arcu ut est posuere malesuada. Donec aliquet, mauris vel viverra bibendum, risus sem fringilla orci, placerat laoreet felis velit ac justo. Mauris sit amet sollicitudin magna. Sed commodo enim sed nibh consectetur cursus. Duis turpis lacus, semper non facilisis eu, semper eu lacus. Donec vel urna urna, eget gravida magna.

Donec purus ipsum, tincidunt sit amet sagittis varius, sollicitudin ac ipsum. Phasellus quam tortor, volutpat nec interdum a, tristique sed turpis. Aenean fringilla, libero in pretium rhoncus, augue nisi sodales libero, at varius quam ipsum feugiat quam. Vestibulum pharetra pellentesque justo, a scelerisque justo varius ultrices. Nam libero augue, ultricies elementum dignissim nec, tincidunt id mi. Fusce ac ligula nibh, vel molestie metus. 

\section{Secção 3} \label{sec3}
Aliquam et sapien at augue tempus congue in ac justo. Donec vehicula tempor mi venenatis dictum. In magna mauris, varius vel sollicitudin ac, lobortis et nunc. Suspendisse nec ultrices leo. Proin vehicula imperdiet neque vitae aliquam. Fusce tincidunt mauris sit amet nulla iaculis ac vulputate augue ornare. Praesent quam eros, suscipit ut pulvinar tristique, dapibus vel turpis. Proin commodo pharetra nisl vitae cursus. Cum sociis natoque penatibus et magnis dis parturient montes, nascetur ridiculus mus. Integer eu metus in turpis lobortis blandit. Vivamus euismod rutrum molestie. Morbi luctus orci tempus enim vestibulum facilisis. Etiam dapibus quam id lorem convallis scelerisque. Fusce tristique enim nec ipsum lacinia pretium.

\subsection{Subsecção}

Nam placerat ullamcorper ante non venenatis. Phasellus et ipsum at lorem rhoncus euismod. Phasellus in risus elit, sed mollis dolor. Aenean non ligula ut metus porta laoreet. Duis mi quam, sollicitudin non posuere eu, facilisis vestibulum purus. Cras eget odio et diam imperdiet consectetur eu vel libero. Cras in dapibus felis. Praesent sed nunc neque. Donec lobortis venenatis pretium. Praesent quis lorem ipsum, id mattis ante. 



\include{Exemplos}





%% Fim da inserção dos capitulos


%% Inicio Bibliografia
\cleardoublepage
\phantomsection
\addcontentsline{toc}{chapter}{Bibliografia}
%%%%%%%%%%%%%%%%
% Escolher entre as duas opcções
%
% A primeira é a aconselhada pelo despacho reitoral
% A segunda é a utilizada pelo IEEE
%
%Primeira opcção
\bibliographystyle{estilo-biblio}				%Estilo bibliografia com nomes
\bibliography{bibliografia}					%Entrada biblbiografia aconselhada com nomes
%
% Segunda opcção
%\bibliographystyle{IEEEtran}					%Estilo bibliografia IEEE
%\bibliography{IEEEabrv,bibliografia}				%Entrada bibliografia aconselhada para IEEE
%% Fim Bibliografia


%%Anexos
\appendix
 
\include{Anexos}
\cleardoublepage


%%Glossário
\newpage
\section*{\titulos{Glossário}}
\vspace{0.5cm}
	\noindent\begin{tabularx}{\linewidth}{l p{0.5cm} Y}
	\LaTeX & & Conjunto de macros para o processador de textos \TeX, utilizado amplamente para a produção de textos matemáticos e científicos devido à sua alta qualidade tipográfica.\cr
	\end{tabularx}
\cleardoublepage



%%Inserir índice remissivo
\printindex

\end{document}
