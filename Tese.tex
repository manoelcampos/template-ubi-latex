\documentclass[10pt,twoside]{template/estiloUBI}

\usepackage{template/formatacaoUBI}

\usepackage{array,verbatim}
\setmainfont{Trebuchet MS}

%%Comentar a linha seguinte se escrever a tese em inglês
\portugues

%%Para gerar índice remissivo (que será inserido no documento com o comando \printindex)
\makeindex

\thesistitle{Título da Tese}

% Se existir um sub-título, descomente a linha abaixo e 
%\thesissubtitle{Sub-título da Tese}

\thesisauthors{Nome do Autor da Tese}
\thesistype{Tese de Doutoramento}
\thesislocalanddate{Covilhã, May of 2016.}

\thesissupervisors{
	Orientador: Prof. Doutor Nome\\
	Co-orientador: Prof. Doutor Nome\\
}

\thesiscourse{Nome do seu Curso}

% Escolher tipo de letra a usar:
% \usepackage{lmodern} 	%Latin modern
% \usepackage{palatino} 	%Palatino
% \usepackage{times} 	%Times
% \renewcommand{\rmdefault}{trebuchet} 	%Trebuchet (caso esteja instalado)

% O comando seguinte insere o nome da tese no cabeçalho das páginas (comentar se não for pretendido)
% (Pode ser o título abreviado)
\cabecalho{\thesistitlestr}

\begin{document}

% Se desejar colocar o conteúdo de uma secção em um ficheiro tex separado,
% tal ficheiro pode ser incluído usando o comando \input{Nome do ficheiro criado}.
% Exemplo: \thesisdedication{\input{nome-do-ficheiro-contendo-a-dedicatoria}}
% Isto resulta para todas as seções abaixo.
\thesisdedication{
	Inserir dedicatória. (Opcional: se não desejar incluir, todo o comando deve ser apagado)\\
}

\thesisthanks{
	Agradecer a quem de direito. (Opcional: se não desejar incluir, todo o comando deve ser apagado)
}

\thesisforewords{
	Prefácio. (Opcional: se não desejar incluir, todo o comando deve ser apagado)
}

\thesisresumo{
	Resumo do trabalho em português, seguida das palavras-chave.
}{Suas, palavras, chaves, separadas, por, vírgula} 	


\thesisresumoalargado{
	Resumo alargado deve ser escrito em português e é usado unicamente para 
	teses escritas em língua estrangeira.
	Se não for este o caso, todo o comando deve ser apagado.
} 	

\thesisabstract{
	Abstract in English, followed by keywords.
}{Your, key, words, separated, by, comma}	

\tableofcontents   % Índice
\listoffigures     % Lista de Figuras    (Opcional)
\listoftables      % Lista de Tabelas    (Opcional)
\listofalgorithms  % Lista de Algoritmos (Opcional)
\thesisacronyms{\input{Acronimos}}  % Lista de Acrónimos  (Opcional)

\mainmatter

% Os capítulos são inseridos a partir daqui 
\chapter{Introdução} \label{chap:int}

%% Para fazer um mini indice do capitulo abrir o ficheiro ``formatacaoUBI.tex" e procurar ''%% O código seguinte permite gerar um mini indice de capitulo (não referido no despacho reitoral)''
%\minitoc


\section{Objectivos}
Este documento pretende servir como modelo\index{modelo} para teses a apresentar na Universidade da Beira Interior (UBI). Para mais informações sobre o {\LaTeX} pode consultar \cite{short} ou \cite{eprojects}.

\section{Secção 2} \label{sec2}
Lorem ipsum dolor sit amet, consectetur adipiscing elit. Praesent at magna viverra neque bibendum pellentesque. Morbi ullamcorper auctor turpis vitae mollis. Fusce elementum mauris eu magna tristique vel aliquet erat iaculis. Donec sed augue mi. Aenean commodo lorem ac nulla iaculis rhoncus. Mauris facilisis, ante in molestie bibendum, lorem augue vehicula metus, ac auctor turpis quam nec purus. Nam malesuada accumsan neque, quis vulputate nibh dapibus vitae. Vestibulum eu arcu ut est posuere malesuada. Donec aliquet, mauris vel viverra bibendum, risus sem fringilla orci, placerat laoreet felis velit ac justo. Mauris sit amet sollicitudin magna. Sed commodo enim sed nibh consectetur cursus. Duis turpis lacus, semper non facilisis eu, semper eu lacus. Donec vel urna urna, eget gravida magna.

Donec purus ipsum, tincidunt sit amet sagittis varius, sollicitudin ac ipsum. Phasellus quam tortor, volutpat nec interdum a, tristique sed turpis. Aenean fringilla, libero in pretium rhoncus, augue nisi sodales libero, at varius quam ipsum feugiat quam. Vestibulum pharetra pellentesque justo, a scelerisque justo varius ultrices. Nam libero augue, ultricies elementum dignissim nec, tincidunt id mi. Fusce ac ligula nibh, vel molestie metus. 

\section{Secção 3} \label{sec3}
Aliquam et sapien at augue tempus congue in ac justo. Donec vehicula tempor mi venenatis dictum. In magna mauris, varius vel sollicitudin ac, lobortis et nunc. Suspendisse nec ultrices leo. Proin vehicula imperdiet neque vitae aliquam. Fusce tincidunt mauris sit amet nulla iaculis ac vulputate augue ornare. Praesent quam eros, suscipit ut pulvinar tristique, dapibus vel turpis. Proin commodo pharetra nisl vitae cursus. Cum sociis natoque penatibus et magnis dis parturient montes, nascetur ridiculus mus. Integer eu metus in turpis lobortis blandit. Vivamus euismod rutrum molestie. Morbi luctus orci tempus enim vestibulum facilisis. Etiam dapibus quam id lorem convallis scelerisque. Fusce tristique enim nec ipsum lacinia pretium.

\subsection{Subsecção}

Nam placerat ullamcorper ante non venenatis. Phasellus et ipsum at lorem rhoncus euismod. Phasellus in risus elit, sed mollis dolor. Aenean non ligula ut metus porta laoreet. Duis mi quam, sollicitudin non posuere eu, facilisis vestibulum purus. Cras eget odio et diam imperdiet consectetur eu vel libero. Cras in dapibus felis. Praesent sed nunc neque. Donec lobortis venenatis pretium. Praesent quis lorem ipsum, id mattis ante. 


\include{Exemplos}
% Fim da inserção dos capítulos

% Bibliografia
% Primeiro parâmetro é o estilo e o segundo o arquivo bib
\thesisbibliography{template/estilo-biblio}{bibliografia}

% Exemplo de uso de outro estilo bibliográfico. Define ser definido apenas um estilo 
%\thesisbibliography{template/IEEEtran}{references}

\appendix{\input{Anexos}} % Anexos (Opcional)
\thesisglossary{\begin{tabularx}{\linewidth}{l p{0.5cm} Y}
	\LaTeX & & Conjunto de macros para o processador de textos \TeX, utilizado amplamente para a produção de textos matemáticos e científicos devido à sua alta qualidade tipográfica.\cr
\end{tabularx}}  % Glossário (Opcional)

\printindex % Inserir índice remissivo (Opcional)

\end{document}
