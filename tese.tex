\documentclass[10pt,twoside]{./template/StyleUBI}

\usepackage{./template/FormattingUBI}

\usepackage{array,verbatim}
\setmainfont{Trebuchet MS}

%%Comentar a linha seguinte se escrever a tese em inglês
\portugues

%%Para índice remissivo
\makeindex

\thesistitle{Título da Tese}
%\thesissubtitle{An comprehensive approach}
\thesisauthors{Nome do autor}
\thesistype{PhD Thesis Proposal}
\thesislocalanddate{Covilhã, May of 2016.}
\thesissupervisors{
	Supervisor: Prof. Doutor Nome\\
	%Orientador: Prof. PhD Nome\\
	Co-supervisor: Prof. Doutor Nome\\
}
\thesiscourse{Engenharia Informática}

%%Escolher tipo de letra a usar:
%\usepackage{lmodern} 	%Latin modern
%\usepackage{palatino} 	%Palatino
%\usepackage{times} 	%Times
%\renewcommand{\rmdefault}{trebuchet} 	%Trebuchet (caso esteja instalado)


%%O comando seguinte insere o nome da tese no cabeçalho das páginas (comentar se não for pretendido)
%(Pode ser um título abreviado)
\cabecalho{\thesistitlestr}

\begin{document}
%%Dedicatória (se não desejar incluír dedicatório, todo o comando deve ser apagado)
%Isto vale para todas as seções abaixo.
\thesisdedication{
    %Se desejar colocar o conteúdo em um ficheiro tex separado,
    %ele pode ser incluído usando o comando \input{Nome do ficheiro criado}.
    %Isto resulta para todas as seções abaixo.
	Inserir dedicatória (opcional)
}

%%Agradecimentos 
\thesisthanks{
	Agradecer a quem de direito (opcional)
}

%%Prefácio
\thesisforewords{
	Prefácio (Opcional)
}

%%Resumo+palavras-chave (em português)
\thesisresumo{
	Resumo do trabalho
}
{Suas, palavras, chaves, separadas, por, vírgula} 	


%%Resumo alargado (em português)
\thesisresumoalargado{
	Unicamente para teses em língua estrangeira
} 	

%%abstract+keywords
\thesisabstract{
	Abstract in English
}
{Your, key, words, separated, by, comma}	


\tableofcontents %Índice
\listoffigures   %Lista de figuras
\listoftables    %Lista de tabelas

%Lista de Algoritmos (algorithms list)
\newpage
\phantomsection \label{listofalgs}
\addcontentsline{toc}{chapter\label{listofalgs}}{List of Algorithms}
\listofalgorithms
\cleardoublepage


%%Abreviaturas
\phantomsection \label{listofabbreviations}
\addcontentsline{toc}{chapter\label{listofabbreviations}}{List of Abbreviations}
%%Abreviaturas
\newpage
\section*{\titulos{Lista de Acrónimos}}
\vspace{0.5cm}
  \begin{tabularx}{\linewidth}{c p{0.5cm} Y}
 	UBI & & Universidade da Beira Interior\cr
 	MPSOCD & & Multi-objective Particle Swarm Optimization Crowding Distance
 	\end{tabularx}
 \cleardoublepage


\mainmatter

%% Os capitulos são inseridos a partir daqui 
\include{introducao}
\include{exemplos}
%% Fim da inserção dos capitulos

%Bibliografia (bibliography)
%Primeiro parâmetro é o estilo e o segundo o arquivo bib
\thesisbibliography{./template/BibliographyStyle}{bibliografia}
%\thesisbibliography{./template/IEEEtran}{references}

%%Anexos
\appendix{
	\chapter{Anexos}
\label{chap:anexos}

\section{Datasheets dos componentes utilizados}


}

%%Glossário
\thesisglossary{
	\begin{tabularx}{\linewidth}{l p{0.5cm} Y}
	\LaTeX & & Conjunto de macros para o processador de textos \TeX, utilizado amplamente para a produção de textos matemáticos e científicos devido à sua alta qualidade tipográfica.\cr
	\end{tabularx}
}

%%Inserir índice remissivo
\printindex

\end{document}
